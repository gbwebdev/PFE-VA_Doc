Nous avons bien conscience que ce projet peut sembler relativement ambitieux, notamment au regard de l’estimation de la charge de travail qu’il représente.\\
Il s’agit cependant d’un projet que nous maîtrisons dans les moindres détails et pour lequel nous éprouvons une motivation sans limite (et dont peuvent témoigner les nuits déjà passées à la préparation de cette étude préliminaire). C’est pourquoi nous avons une absolue confiance quant à sa faisabilité (tout en étant conscient des difficultés que cela représente).\\

Nous sommes en outre intimement persuadés que la plate forme faisant l’objet de ce projet pourrait être d’un grand bénéfice à l’IPSA, et ce à bien des niveaux : elle pourrait servir de “fil rouge” et de support de TP pour toutes les matières orientées “systèmes”, servir de base à divers projets de recherches, mais aussi faire office de démonstrateur de bon nombre de compétences développées à l’IPSA.\\

Enfin, les possibilités d’évolution de la plateforme sont innombrables : ajout d’une capacité d’interaction des robots via le réseau afin d’optimiser leurs déplacements, d’une capacité de mémorisation afin qu’ils puissent dresser une carte de leur environnement pour s’y déplacer plus efficacement, approfondissement de l’utilisation de la reconnaissance d’image afin de s’en servir pour évaluer les distances, détecter des objets étrangers et inconnus, implémentation d’un serveur web sur les robots afin de pouvoir “prendre la main”en temps réel depuis un ordinateur (pour les piloter manuellement, ou appliquer un stimuli sur leur comportement normal, par exemple), navigation “à vue”, se passant de la ligne blanche etc…\\

Nous espérons réellement pouvoir concretiser ce projet en tant que PFE, mais aurons en tout les cas bénéficié d’une expérience plus qu’enrichissante dans le cadre de la préparation de ce dossier, qui nous a permis de couvrir un très large champ d’application de nos “nouvelles compétences” (aussi bien techniques que liées à la gestion de projet) sur un sujet passionant et en autonomie. 