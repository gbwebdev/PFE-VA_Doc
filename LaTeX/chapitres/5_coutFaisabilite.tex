 Nous avons fait notre possible pour estimer les coûts liés à ce projet. Il s'agit évidemment d'un exercice complexe.
 Il est toujours plus simple d'évaluer le coût matériel que le coût humain. Ici, la main d'oeuvre est gratuite, mais il n'empêche qu'un planning réaliste serait un atout de taille pour mener ce projet à bien.

 \subsection{Fournitures}

 	Le descriptif technique détaillé élaboré au cours de ce dossier nous permet de dresser une liste précise et sans doute proche de l'exhaustivité du matériel nécessaire.\\

 	Celà nous a permis de définir les coûts suivants :
 	\begin{itemize}
 		\item Coût d'un robot : 270€ HT
 		\item Coût du circuit (hors impression 3D): 85€ HT
 		\item Coût du chargeur : 20€ HT
 		\item Coût des composants divers : 10€
 	\end{itemize}

 	\vspace{15pt}

 	Soit, pour un ensemble d'un circuit et deux robots, un total de \textbf{450€ HT} (525€ TTC).\\

 	Le détail des calcul est disponible en annexe \ref{devis} (page \pageref{devis}).



 	Nous avons sélectionné les principaux fournisseurs en fonction de la compatibilité de leur catalogue avec notre besoin.
 	Nous avons essayé de limiter le nombre de fournisseurs afin de simplifier la "logistique" et limiter les frais de port.\\

 	Heureusement, les fournisseurs sur Internet disposent aujourd'hui de très larges catalogues et nous n'avons pas été limités dans notre choix de composants.

 \subsection{Volume de travail}

 	Nous avons identifié plusieurs « work packages » :\\
 	\begin{itemize}
 		\item Fournitures
 		\item Réalisation des cartes
 		\item Assemblage du robot
 		\item Réalisation du circuit
 		\item Développement logiciel
 	\end{itemize}

 	\newpage

	Les tâches du projet ont été définies, estimées et réparties dans les « work packages » comme ceci :\\

	\illu{planning.pdf}{Estimation du planning}{0.87}

	\vspace{15pt}

	Ceci nous a permis d'établir les diagrammes de Gantt et de PERT, et donc d’en déduire un processus, une marche à suivre, pour la réalisation du projet.\\

	\illu{Gantt.png}{Diagramme de Gantt du projet (agrandissement disponible en annexe \ref{gantt})}{0.3}
	\vspace{15pt}
	Le chemin critique est représenté par les cases hachurées.\\

	Du diagramme de Gantt, nous avons pu déduire le diagramme PERT suivant:
	\illu{PERT1.png}{Diagramme PERT du projet (1/2) (agrandissement disponible en annexe \ref{PERT})}{0.6}
	\illu{PERT2.png}{Diagramme PERT du projet (2/2) (agrandissement disponible en annexe \ref{PERT})}{0.4}
	\vspace{15pt}
	Le chemin critique est représenté par les cellules en jaune.\\

	Ces diagrammes ont été réalisés en se basant sur l’hypothèse d'une disponibilité totale des ressources et de l'absence absolue d'imprévus. S'agissant évidemment d'une hypothèse parfaitement irréaliste, et malgré le "pessimisme" dont nous avons volontairement fait preuve pour estimer les volumes horaires, nous envisageons une majoration de la durée du projet et de la quantité de travail de l’ordre de 20\%.